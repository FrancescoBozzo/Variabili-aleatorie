\documentclass[
	,a4paper
	,12pt
	,oneside
]{article}

\usepackage{style}

\title{\textsc{Riassunto variabili aleatorie}}
\author{\textsc{Francesco Bozzo}}
\date{\today}

\newtheorem*{theorem}{Teorema}

\begin{document}

\maketitle

\begin{aleatoria}{bernoulliana}{Ber(\(p\))}
	Si utilizza per definire un singolo esperimento aleatorio. X rappresenta il successo o il fallimento, che possono rispettivamente avvenire con probabilità \(p\) e \(1-p\).
	\probabilita{
		\begin{cases}
			1-p & \text{per } x=0\\
			p & \text{per } x=1
		\end{cases}
	}

	\speranza{p}
	\varianza{p(1-p)}
\end{aleatoria}

\begin{aleatoria}{geometrica}{Geo(\(p\))}
	X descrive il numero totale di prove \(n\) necessarie prima che avvenga il primo successo. Ogni prova è stocasticamente indipendente dalle altre e la probabiltà \(p\) che avvenga il singolo successo è costante.
	\probabilita{
		\begin{cases}
			p(1-p)^{x-1} & \text{per } x=1,2,3,\dots\\
			0 & \text{altrove }
		\end{cases}
	}
	\distribuzione[1-(1-0.5)^floor(x)]{1-(1-p)^{[x]}}

	\begin{theorem}[assenza di memoria]
		\(\text{Pr}(T=m+n|T>m) = \text{Pr}(T=n)\)
	\end{theorem}
	\speranza{\frac{1}{p}}
	\varianza{\frac{1-p}{p^2}}
\end{aleatoria}

\begin{aleatoria}{binomiale}{Bin(\(n, p\))}
	X rappresenta la variabile che descrive il numero di successi \(x\) su \(n\) prove totali, con probabilità \(p\).
	\probabilita{\binom{n}{x}p^x(1-p)^{n-x}}

	\speranza{np}
	\varianza{np(1-p)}
\end{aleatoria}

\begin{aleatoria}{binomiale negativa}{NeBin(\(r, p\))}
	X descrive il numero totale di prove affinché avvengano \(r\) successi, ciascuno con probabilità \(p\).
	\probabilita{
		\begin{cases}
			\binom{x-1}{r-1}p^r(1-p)^{x-r} & \text{per } x=r, r+1, \text{\dots}\\
			0 & \text{altrimenti}
		\end{cases}
	}

	\begin{theorem}
		\(X\sim\text{NeBin}(r,p)\text{, } Z\sim\text{Bin}(n,p) \implies \text{Pr}(Z\geqslant r) = \text{Pr}(X\leqslant n)\)
	\end{theorem}
	\speranza{\frac{n}{p}}
	\varianza{n\frac{1-p}{p^2}}
\end{aleatoria}

\begin{aleatoria}{di Poisson}{Poisson(\(\lambda\))}
	X descrive il numero di successi di un evento, quando il numero prove \(m\) tende ad infinito ed il prodotto \(mp\) è una costante. Ovvero mediamente si verificano un numero di successi \(\lambda\).
	\probabilita{\frac{\lambda^x}{x!}e^{-\lambda}}

	\speranza{\lambda}
	\varianza{\lambda}
\end{aleatoria}

\begin{aleatoria}{di Gauss o normale}{N(\(\mu, \sigma\))}
	Si definisce \(\mu\) il parametro di \emph{posizione} e \(\sigma\) il parametro di \emph{scala}. Si noti che non esiste un'espressione analitica per la funzione di distribuzione (si usino le \emph{tavole di Sheppard}).
	\densita[(3/sqrt(2*3.14))*exp(-0.5*(x-5)^2)]{\frac{1}{\sqrt{2\pi\sigma}}e^{-\frac{(x-\mu)^2}{2\sigma^2}}}

	\begin{theorem}[standardizzazione delle variabili aleatorie]
		\(X\sim\text{N}(\mu,\sigma)\implies z:=\frac{X-\mu}{\sigma}\sim\text{N}(0,1)\)
	\end{theorem}
	\begin{theorem}
		\(Z\sim\text{N}(0,1)\text{, }\mu\in\mathbb{R}\text{, }\sigma\in\mathbb{R}^+\implies X=\sigma Z+\mu\sim\text{N}(\mu,\sigma)\)
	\end{theorem}
	\begin{theorem}
		\(X\sim\text{N}(\mu,\sigma)\implies Y=a+bX\sim\text{N}(a+b\mu,b^2\sigma)\)
	\end{theorem}
	\speranza{\mu}
	\varianza{\sigma^2}
\end{aleatoria}

\begin{aleatoria}{esponenziale}{Exp(\(\lambda\))}
	\densita[1*exp(-1*x)]{
	\begin{cases}
		\lambda e^{-\lambda x} & \text{per } x \geqslant 0 \\
		0 & \text{altrimenti}
	\end{cases}
	}
	\distribuzione[1-exp(-1*x)]{
		\begin{cases}
			1 - e^{-\lambda x} & \text{per } x \geqslant 0 \\
			0 & \text{altrimenti}
		\end{cases}
	}

	\begin{theorem}[assenza di memoria]
		\(\text{Pr}(X>a+b|X>a) = \text{Pr}(X>b)\)
	\end{theorem}
	\speranza{\frac{1}{\lambda}}
	\varianza{\frac{1}{\lambda^2}}
\end{aleatoria}

\begin{aleatoria}{uniforme}{Unif(\(a, b\))}
	\densita{
		\begin{cases}
			\frac{1}{b-a} & \text{per } x \in (a, b] \\
			0 & \text{altrimenti}
		\end{cases}
	}

	\speranza{\frac{1}{b-a}}
	\varianza{\frac{(b-a)^2}{12}}
\end{aleatoria}

\end{document}
