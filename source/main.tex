\documentclass[
	,a4paper
	,12pt
	,oneside
]{article}

\usepackage{style}

\title{\textsc{Riassunto variabili aleatorie}}
\author{\textsc{Francesco Bozzo}}
\date{\today}

\begin{document}

\maketitle

\begin{aleatoria}{bernoulliana}{Ber(\(p\))}
	Si utilizza per definire un singolo esperimento aleatorio.
	\densita{
		\begin{cases}
			1-p & \text{per } x=0\\
			p & \text{per } x=1
		\end{cases}
	}
\end{aleatoria}

\begin{aleatoria}{geometrica}{Geo(\(n, p\))}
	\distribuzione[1-(1-0.5)^floor(x)]{1-(1-p)^{[x]}}
\end{aleatoria}

\begin{aleatoria}{binomiale}{Bin(n, p)}
	\densita[]{\binom{n}{k}p^k(1-p)^{n-k}}
\end{aleatoria}

\begin{aleatoria}{binomiale negativa}{NeBin(\(r, p\))}
	\densita{
		\begin{cases}
			\binom{x-1}{r-1}p^r(1-p)^{x-r} & \text{per } x=r, r+1, \text{\dots}\\
			0 & \text{altrimenti}
		\end{cases}
	}
\end{aleatoria}

\begin{aleatoria}{di Poisson}{Poisson(\(\lambda\))}
	\densita{\frac{\lambda^k}{k!}e^{-\lambda}}
\end{aleatoria}

\begin{aleatoria}{di Gauss o normale}{N(\(\mu, \sigma\))}
	\densita[(3/sqrt(2*3.14))*exp(-0.5*(x-5)^2)]{\frac{1}{\sqrt{2\pi\sigma}}e^{-\frac{(x-\mu)^2}{2\sigma^2}}}
\end{aleatoria}

\begin{aleatoria}{esponenziale}{Exp(\(\lambda\))}
	\densita[1*exp(-1*x)]{
	\begin{cases}
		\lambda e^{-\lambda x} & \text{per } x \geqslant 0 \\
		0 & \text{altrimenti}
	\end{cases}
	}
	\distribuzione[1-exp(-1*x)]{
		\begin{cases}
			1 - e^{-\lambda x} & \text{per } x \geqslant 0 \\
			0 & \text{altrimenti}
		\end{cases}
	}
\end{aleatoria}

\begin{aleatoria}{uniforme}{Unif(\(a, b\))}
	\densita{
		\begin{cases}
			\frac{1}{b-a} & \text{per } x \in (a, b] \\
			0 & \text{altrimenti}
		\end{cases}
	}
\end{aleatoria}

\end{document}
