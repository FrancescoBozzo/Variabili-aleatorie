\documentclass[
	,a4paper
	,12pt
	,oneside
]{article}

\usepackage{style}

\title{\textsc{Riassunto variabili aleatorie}}
\author{\textsc{Francesco Bozzo}}
\date{\today}

\begin{document}

\maketitle

\begin{aleatoria}{bernoulliana}{Ber}
	Si utilizza per definire un singolo esperimento aleatorio.
	\densita{
		\begin{cases}
			1-p & \text{per } x=0\\
			p & \text{per } x=1
		\end{cases}
	}
\end{aleatoria}

\begin{aleatoria}{geometrica}{Geo}
	\distribuzione[1-(1-0.5)^floor(x)]{1-(1-p)^{[x]}}
\end{aleatoria}

\begin{aleatoria}{binomiale}{Bin}

\end{aleatoria}

\begin{aleatoria}{binomiale negativa}{BiNeg}
\end{aleatoria}

\begin{aleatoria}{di Poisson}{Poi}
\end{aleatoria}

\begin{aleatoria}{di Gauss o normale}{Gau}
\end{aleatoria}

\begin{aleatoria}{esponenziale}{Esp}
\end{aleatoria}

\begin{aleatoria}{uniforme}{Unif}
\end{aleatoria}

\end{document}
